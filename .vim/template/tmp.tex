\documentclass[dvipdfmx,uplatex,titlepage]{jsarticle}
\usepackage[dvipdfmx]{graphicx}
\usepackage{bm}
\usepackage{multicol}
\usepackage{here}
\usepackage{amsmath,amssymb}
\usepackage{indentfirst}
\usepackage{url}
\usepackage{ascmac}
\usepackage[hang,small,bf]{caption}
\usepackage[subrefformat=parens]{subcaption}
\captionsetup{compatibility=false}


\title{template}
\author{}
\date{\today}


\begin{document}
\maketitle


\begin{eqnarray}
\end{eqnarray}


\begin{figure}[H]
    \centering
    \includegraphics[width=8cm]{mat_vec_n_384.pdf}
    \captionsetup{labelformat=empty,labelsep=none}
    \caption{行列の大きさ384}
\end{figure}

\begin{lstlisting}[basicstyle=\ttfamily\footnotesize, frame=single]
#include<stdio.h>

#define N 12000

int main(){
    int i;
    int a=13;
    int b=5;
    int m=24;
    int x[N];

    x[0] = 8;

    for(i=1;i<N;i++){
        x[i]=( a * x[i-1] + b ) % m;
    }

    for(i=0;i<N;i=i+3){
        printf("%d %d %d \n",x[i],x[i+1],x[i+2]);
    }

}
\end{lstlisting}


\begin{itembox}[l]{ベイズの定理}
    \begin{eqnarray}
        \label{eq32}
        P( \omega _i | x_t) &=& \frac { P( \omega _i , x_t ) } {P(x_t)} \nonumber\\
         &=& \frac { P( \omega _i ) \cdot P(x_t| \omega _i)} {P(x_t)} \nonumber\\
                            &=& \frac { \pi _i  \cdot P(x_t| \omega _i)} {P(x_t)}
    \end{eqnarray}
\end{itembox}

\end{document}


